% German style date formatting (footer)
\usepackage[ddmmyyyy]{datetime}
\renewcommand{\dateseparator}{.}

% Format the captions used for figures etc.
\usepackage[compatibility=false]{caption}
\captionsetup{singlelinecheck=off,justification=raggedleft,labelformat=empty,labelsep=none}



\usepackage{etex} % cures ``No room for a new \dimen'' error

%%%%%% font size handling %%%%%%
\RequirePackage{fix-cm}
\usepackage{lmodern}
\DeclareFontFamily{OMX}{lmex}{} % proper math mode size (http://tex.stackexchange.com/questions/74623/big-integral-in-lmodern)
\DeclareFontShape{OMX}{lmex}{m}{n}{<-> lmex10}{}
% \addtokomafont{disposition}{\rmfamily}

%%%%%% better enumerate environment
\usepackage{enumitem}

%%%%%% bold math %%%%% (boldsymbol)
\usepackage{bm}

%%%%%% textdegree %%%%%
\usepackage{ textcomp }

%%%%%% tikz %%%%%%
% \usepackage{pdfpages}
\usepackage{pgfplots}
\pgfplotsset{compat=newest}
\usepackage{tikz}
\usetikzlibrary{fit}

\usetikzlibrary{arrows}
\usetikzlibrary{arrows.meta} % propper scaling of arrows
\pgfplotsset{plot coordinates/math parser=false}
\usetikzlibrary{plotmarks}
% \usetikzlibrary{decorations.shapes}
% \usetikzlibrary{decorations.pathmorphing}
\usepgfplotslibrary{fillbetween}
\usetikzlibrary{patterns}
\usetikzlibrary{intersections}
% \usetikzlibrary{decorations.markings}
% \usetikzlibrary{arrows.meta}
\usepackage{graphicx}
\usepackage{amssymb}
\usepackage{multimedia}
\usepackage{array}
\usepackage{booktabs,adjustbox}
\usepackage{marvosym}
\usepackage{hyperref}
\usetikzlibrary{calc}
% \usepackage{animate}
% \usepackage{movie15}
\usepackage{pdfpages}
\usepackage{pifont}
\newcommand{\cmark}{\ding{51}}%
\usepackage[thinlines]{easytable}

%%%%%% additional packages %%%%%%
% \usepackage{wrapfig}
\usepackage[absolute,overlay]{textpos}
\usetikzlibrary{positioning}
\usepackage{arydshln}
\usepackage{setspace}
\usepackage{multirow}
\usepackage{multicol}
\usepackage{import}
\usepackage{mathtools}

\usepackage{diagbox}
\usetikzlibrary{matrix}


%%%%%% plot sizes %%%%%%
\def\plottitlefontsize{\Large}
\def\plotlabelfontsize{\normalsize}
\def\plotlegendfontsize{\scriptsize}
\def\plottickfontsize{\normalsize}
% \def\plottickfontsize{\tiny}
\def\plotlinewidth{1pt}
\def\plotnodefontsize{\small}

\newlength\figureheight
\newlength\figurewidth

%%%%%% triangles as item markers %%%%%%
\setlist[itemize,1]{label={\small\raise2.5pt\hbox{\donotcoloroutermaths{\color{rwth-75}$\blacktriangleright$}}}}
\setlist[itemize,2]{label={\small\raise2.5pt\hbox{\donotcoloroutermaths{\color{rwth}$\triangleright$}}}}




 \usepackage{pdfpages}
%%%%%% a footnote box for references %%%%%%
\newcounter{grcitenumber}
\newcommand{\declcite}[1]{\refstepcounter{grcitenumber}\label{#1}}
% \newcommand{\showcite}[1]{\ensuremath{\grm{^{\mathit{[\ref{#1}]}}}}}
\newcommand{\showcite}[1]{\ensuremath{\grm{^{\mathit{[#1]}}}}}
% % \newcommand{\greycite}{\stepcounter{grcitenumber}{\ensuremath{\bm{^{\mathit{[\thegrcitenumber]}}}}}}
\newcommand{\citetext}[2]{\grm{$\mathit{[#1]:}$\ \textit{#2}}}
\newcommand{\showlargecite}[1]{\grm{$\mathit{[#1]}$}}
\newcommand{\citebox}[2]{\begin{textblock*}{\textwidth}(0.8cm,16.2cm)\setstretch{0.5}\citetext{#1}{#2}\end{textblock*}}
\newcommand{\citeboxx}[4]{\begin{textblock*}{\textwidth}(0.8cm,15.7cm)\setstretch{0.5}\citetext{#1}{#2} \\ \citetext{#3}{#4}  \end{textblock*}}
\newcommand{\citeboxxx}[6]{\begin{textblock*}{\textwidth}(0.8cm,15.2cm)\setstretch{0.5}\citetext{#1}{#2} \\ \citetext{#3}{#4} \\ \citetext{#5}{#6} \end{textblock*}}
\newcommand{\citeboxxxx}[8]{\begin{textblock*}{\textwidth}(0.8cm,14.7cm)\setstretch{0.5}\citetext{#1}{#2} \\ \citetext{#3}{#4} \\ \citetext{#5}{#6} \\ \citetext{#7}{#8} \end{textblock*}}

\def\grm#1{\color{black-75}{#1}\color{black}}
%%%%% theorems %%%%%%
\newtheorem{remark}[theorem]{Remark}

%%%%%% backup slides not counting to page numbers %%%%%%
\newcommand{\backupbegin}{
   \newcounter{finalframe}
   \setcounter{finalframe}{\value{framenumber}}
}
\newcommand{\backupend}{
   \setcounter{framenumber}{\value{finalframe}}
}

%%%%% custom commands %%%%%%
\def\rhf#1{#1 \rfloor} 
\def\prhf#1#2{#1 \rfloor_{#2}} 
\def\lhf#1{\lceil #1}
\def\plhf#1#2{\lceil^{#2} #1}
\def\mean#1{\langle #1 \rangle}
\def\ns#1{\underline{#1}}
\def\ssum#1{\sum_{\substack{#1}}}
\def\sprod#1{\prod_{\substack{#1}}}
\def\ltimes{\times \ldots \times}
\def\lotimes{\otimes \ldots \otimes}
\def\tr#1{{#1}^T}
\def\sp#1#2{\langle #1,#2 \rangle}
\def\sm#1{\mbox{\large\boldmath$($} #1 \mbox{\large\boldmath$)$}}
\def\mean#1{\langle #1 \rangle}
\def\q#1{``#1``}
\def\lhb#1{\mathcal{L}(#1)} 
\def\rhb#1{\mathcal{R}(#1)}

\newcommand{\diver}{\mathop{\rm div}}
\newcommand{\Ind}{\mathcal{I}}
\newcommand{\Index}{\mathcal{I}}
\newcommand{\K}{\ensuremath{\mathbb{K}}}
\newcommand{\R}{\ensuremath{\mathbb{R}}}
\newcommand{\C}{\ensuremath{\mathbb{C}}}
\newcommand{\N}{\ensuremath{\mathbb{N}}}
\newcommand{\Z}{\ensuremath{\mathbb{Z}}}
% \newcommand{\argmin}{\mathop{\rm argmin}}
\DeclareMathOperator*{\argmin}{argmin}
\newcommand*\intd{\mathop{}\!\mathrm{d}}





